\documentclass[]{article}

\usepackage[left=2.00cm, right=2.00cm, top=2.00cm, bottom=2.00cm]{geometry}
\usepackage[spanish,es-noshorthands]{babel}
\usepackage[utf8]{inputenc} % para tildes y ñ
\usepackage{graphicx} % para las figuras
\usepackage{xcolor}
\usepackage{listings} % para el código fuente en c++

\lstdefinestyle{customc}{
  belowcaptionskip=1\baselineskip,
  breaklines=true,
  frame=single,
  xleftmargin=\parindent,
  language=C++,
  showstringspaces=false,
  basicstyle=\footnotesize\ttfamily,
  keywordstyle=\bfseries\color{green!40!black},
  commentstyle=\itshape\color{gray!40!gray},
  identifierstyle=\color{black},
  stringstyle=\color{orange},
}
\lstset{style=customc}


%opening
\title{Práctica 2. Programación dinámica}
\author{AGUSTIN ENEAS HARISPE LUCARELLI \\ % mantenga las dos barras al final de la línea y este comentario
correo@servidor.com \\ % mantenga las dos barras al final de la linea y este comentario
Teléfono: xxxxxxxx \\ % mantenga las dos barras al final de la línea y este comentario
NIF: 55080402W \\ % mantenga las dos barras al final de la línea y este comentario
}


\begin{document}

\maketitle

%\begin{abstract}
%\end{abstract}

% Ejemplo de ecuación a trozos
%
%$f(i,j)=\left\{ 
%  \begin{array}{lcr}
%      i + j & si & i < j \\ % caso 1
%      i + 7 & si & i = 1 \\ % caso 2
%      2 & si & i \geq j     % caso 3
%  \end{array}
%\right.$

\begin{enumerate}
\item Formalice a continuación y describa la función que asigna un determinado valor a cada uno de los tipos de defensas.

La estrategia a seguir en esta práctica para el centro de extracción de minerales se basa en colocarla lo más cerca del centro posible. La función debe darle un mayor valor a la celda que más cerca este del centro del mapa. Donde x es la diferencia entre la posición del mapa y el centro de la celda a evaluar podemos decir que nuestra función es racional, o sea del tipo f(x)=k/x, donde k es un entero positivo para ajustar nuestra escala.
\vspace{5mm}
% Elimine los símbolos de tanto por ciento para descomentar las siguientes instrucciones e incluir una imagen en su respuesta. La mejor ubicación de la imagen será determinada por el compilador de Latex. No tiene por qué situarse a continuación en el fichero en formato pdf resultante.
%\begin{figure}
%\centering
%\includegraphics[width=0.7\linewidth]{./defenseValueCellsHead} % no es necesario especificar la extensión del archivo que contiene la imagen
%\caption{Estrategia devoradora para la mina}
%\label{fig:defenseValueCellsHead}
%\end{figure}


\item Describa la estructura o estructuras necesarias para representar la tabla de subproblemas resueltos.

Necesitamos una matriz de tamaño [nº defensas][ases] para dicha tabla y un dos vectores tamaño [nº defensas] para el valor y costes de las mismas respectivamente.



\item En base a los dos ejercicios anteriores, diseñe un algoritmo que determine el máximo beneficio posible a obtener dada una combinación de defensas y \emph{ases} disponibles. Muestre a continuación el código relevante.

\begin{lstlisting}
void ordenacion_rapida(float *array, int izq, int der, int *indices){
  int i = izq;
  int j = der;
  float aux = *(array + izq); //pivote
  int ind_aux = *(indices + izq);

  if(izq < der){
    while(i<j){
      while(*(array + j) <= aux && i<j){j--;}
      *(array + i) = *(array + j);
       *(indices + i) = *(indices + j);
      while(*(array + i)>= aux && i<j){i++;}
      *(array + j)=*(array + i);
      *(indices + j)=*(indices+ i);
    }
    *(array + i) = aux;
    *(indices + i) = ind_aux;
    ordenacion_rapida(array,izq,i-1,indices);
    ordenacion_rapida(array,j+1,der,indices);
  }
}
\end{lstlisting}

\vspace{5mm}


\item Diseñe un algoritmo que recupere la combinación óptima de defensas a partir del contenido de la tabla de subproblemas resueltos. Muestre a continuación el código relevante.

Podemos ver como el programa sigue el esquema de un algoritmo voraz:
\begin{itemize}

	\item Conjunto de candidatos: celdas.

	\item Función factibilidad: la función factibilidad.

	\item Funcion de selección: extraerValores;

	\item Solución: colocar defensas.

	\item Objetivo: maximizar los segundos que sobrevive el centro de extracción.
	
\end{itemize}


\vspace{5mm}


\end{enumerate}

Todo el material incluido en esta memoria y en los ficheros asociados es de mi autoría o ha sido facilitado por los profesores de la asignatura. Haciendo entrega de este documento confirmo que he leído la normativa de la asignatura, incluido el punto que respecta al uso de material no original.

\end{document}
