Se hacen algunas sencillas pruebas de caja negra, el código no relevante o repetitivo se obvia.

\begin{lstlisting}
void cajanegra(){
  //aqui se realizaran las pruebas con diferentes inputs
  float p02[] = {}; int i02[] = {};
  float p1 = 1;  int i1 = 1; 
  float p2[] = {0,0,0,0,0,0,0}; int i2[] = {0,1,2,3,4,5,6};
  float p3[] = {-1,2,3,4,5,6,7,8}; int i3[] = {0,1,2,3,4,5,6,7}; 
  float p4[] = {8,7,6,5,4,3,2,1}; int i4[] = {0,1,2,3,4,5,6,7}; 
  //[...] mas inicializaciones 
  std::vector<std::pair<float,int>> p44;
  for(int i=0; i<8; i++){
      p44.emplace_back(p4[i],i);
  }

  //por fusion
  merge_sort(&p1,0,0,&i1); //un unico elemento
  printf("%f\n",p1);
  merge_sort(p2,0,6,i2); //elementos repetidos
  for (int i = 0 ; i <*(&p2 + 1) - p2; i++) 
    std::cout << p2[i] << '\t' << i2[i] << std::endl;
  printf("\n");
  merge_sort(p3,0,7,i3); //ordenado ascendentemente
  for (int i = 0 ; i <*(&p3 + 1) - p3; i++) 
    std::cout << p3[i] << '\t' << i3[i] << std::endl;
  printf("\n");
  merge_sort(p4,0,7,i4); //ordenado descendentemente
  for (int i = 0 ; i <*(&p4 + 1) - p4; i++) 
    std::cout << p4[i] << '\t' << i4[i] << std::endl;
  printf("\n\n");

  //rapida
  //[...] igual que fusion
  //monticulo
  ordenar_monticulo(p14); //un unico elemento
  printf("%f %i\n",p14[0].first, p14[0].second);
  ordenar_monticulo(p24);  //elementos repetidos
  for (int i = 0 ; i <7; i++) 
    std::cout << p24[i].first << '\t' << p24[i].second << std::endl;
  printf("\n");
  ordenar_monticulo(p34); //ordenado ascendentemente
  for (int i = 0 ; i <8; i++) 
    std::cout << p34[i].first << '\t' << p34[i].second << std::endl;
  printf("\n");
  ordenar_monticulo(p44); //ordenado descendentemente
  for (int i = 0 ; i <8; i++) 
    std::cout << p44[i].first << '\t' << p44[i].second << std::endl;
  printf("\n\n");

}
\end{lstlisting}

\vspace{5mm}
