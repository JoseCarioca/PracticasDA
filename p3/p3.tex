\documentclass[]{article}

\usepackage[left=2.00cm, right=2.00cm, top=2.00cm, bottom=2.00cm]{geometry}
\usepackage[spanish,es-noshorthands]{babel}
\usepackage[utf8]{inputenc} % para tildes y ñ
\usepackage{graphicx} % para las figuras
\usepackage{xcolor}
\usepackage{listings} % para el código fuente en c++

\lstdefinestyle{customc}{
  belowcaptionskip=1\baselineskip,
  breaklines=true,
  frame=single,
  xleftmargin=\parindent,
  language=C++,
  showstringspaces=false,
  basicstyle=\footnotesize\ttfamily,
  keywordstyle=\bfseries\color{green!40!black},
  commentstyle=\itshape\color{gray!40!gray},
  identifierstyle=\color{black},
  stringstyle=\color{orange},
}
\lstset{style=customc}


%opening
\title{Práctica 3. Divide y vencerás}
\author{AGUSTIN ENEAS HARISPE LUCARELLI \\ % mantenga las dos barras al final de la línea y este comentario
correo@servidor.com \\ % mantenga las dos barras al final de la linea y este comentario
Teléfono: xxxxxxxx \\ % mantenga las dos barras al final de la línea y este comentario
NIF: 55080402W \\ % mantenga las dos barras al final de la línea y este comentario
}


\begin{document}

\maketitle

%\begin{abstract}
%\end{abstract}

% Ejemplo de ecuación a trozos
%
%$f(i,j)=\left\{ 
%  \begin{array}{lcr}
%      i + j & si & i < j \\ % caso 1
%      i + 7 & si & i = 1 \\ % caso 2
%      2 & si & i \geq j     % caso 3
%  \end{array}
%\right.$

\begin{enumerate}
\item Describa las estructuras de datos utilizados en cada caso para la representación del terreno de batalla. 

La estrategia a seguir en esta práctica para el centro de extracción de minerales se basa en colocarla lo más cerca del centro posible. La función debe darle un mayor valor a la celda que más cerca este del centro del mapa. Donde x es la diferencia entre la posición del mapa y el centro de la celda a evaluar podemos decir que nuestra función es racional, o sea del tipo f(x)=k/x, donde k es un entero positivo para ajustar nuestra escala.
\vspace{5mm}
% Elimine los símbolos de tanto por ciento para descomentar las siguientes instrucciones e incluir una imagen en su respuesta. La mejor ubicación de la imagen será determinada por el compilador de Latex. No tiene por qué situarse a continuación en el fichero en formato pdf resultante.
%\begin{figure}
%\centering
%\includegraphics[width=0.7\linewidth]{./defenseValueCellsHead} % no es necesario especificar la extensión del archivo que contiene la imagen
%\caption{Estrategia devoradora para la mina}
%\label{fig:defenseValueCellsHead}
%\end{figure}


\item Implemente su propia versión del algoritmo de ordenación por fusión. Muestre a continuación el código fuente relevante. 

Necesitamos una matriz de tamaño [nº defensas][ases] para dicha tabla y un dos vectores tamaño [nº defensas] para el valor y costes de las mismas respectivamente.




\item Implemente su propia versión del algoritmo de ordenación rápida. Muestre a continuación el código fuente relevante. 

\begin{lstlisting}
void ordenacion_rapida(float *array, int izq, int der, int *indices){
  int i = izq;
  int j = der;
  float aux = *(array + izq); //pivote
  int ind_aux = *(indices + izq);

  if(izq < der){
    while(i<j){
      while(*(array + j) <= aux && i<j){j--;}
      *(array + i) = *(array + j);
       *(indices + i) = *(indices + j);
      while(*(array + i)>= aux && i<j){i++;}
      *(array + j)=*(array + i);
      *(indices + j)=*(indices+ i);
    }
    *(array + i) = aux;
    *(indices + i) = ind_aux;
    ordenacion_rapida(array,izq,i-1,indices);
    ordenacion_rapida(array,j+1,der,indices);
  }
}
\end{lstlisting}

\vspace{5mm}


\item Realice pruebas de caja negra para asegurar el correcto funcionamiento de los algoritmos de ordenación implementados en los ejercicios anteriores. Detalle a continuación el código relevante.

Podemos ver como el programa sigue el esquema de un algoritmo voraz:
\begin{itemize}

	\item Conjunto de candidatos: celdas.

	\item Función factibilidad: la función factibilidad.

	\item Funcion de selección: extraerValores;

	\item Solución: colocar defensas.

	\item Objetivo: maximizar los segundos que sobrevive el centro de extracción.
	
\end{itemize}


\vspace{5mm}


\item Analice de forma teórica la complejidad de las diferentes versiones del algoritmo de colocación de defensas en función de la estructura de representación del terreno de batalla elegida. Comente a continuación los resultados. Suponga un terreno de batalla cuadrado en todos los casos. 

En este caso, una vez colocado el centro de extracción de minerales, la estrategia es homónima a la anterior. El objetivo será minimizar la distancia entre el centro de extracción y las defensas a colocar. Para conseguir esto volvemos a una función racional k/x donde k es un entero positivo que ajusta la escala y x la distancia entre el centro de la celda y la posicion del centro de extracción.


\item Incluya a continuación una gráfica con los resultados obtenidos. Utilice un esquema indirecto de medida (considere un error absoluto de valor 0.01 y un error relativo de valor 0.001). Es recomendable que diseñe y utilice su propio código para la medición de tiempos en lugar de usar la opción \emph{-time-placeDefenses3} del simulador. Considere en su análisis los planetas con códigos 1500, 2500, 3500,..., 10500, al menos. Puede incluir en su análisis otros planetas que considere oportunos para justificar los resultados. Muestre a continuación el código relevante utilizado para la toma de tiempos y la realización de la gráfica.

\begin{lstlisting}
void DEF_LIB_EXPORTED placeDefenses(bool** freeCells, int nCellsWidth, int nCellsHeight, float mapWidth, float mapHeight
              , std::list<Object*> obstacles, std::list<Defense*> defenses) {

    float cellWidth = mapWidth / nCellsWidth;
    float cellHeight = mapHeight / nCellsHeight;

		int N = nCellsWidth*nCellsHeight;
		float** valores = new float* [N];
		for(int i=0; i<N;i++){
			valores[i] = new float[2]; // [][0] = pos [][1] valores cellValue
			valores[i][0] = i;
			valores[i][1] = cellValue(i/nCellsWidth, i%nCellsWidth, freeCells, cellWidth,cellHeight
, mapWidth, mapHeight, obstacles, defenses); // i/ncellswidth=row y i%ncellsheight=col
		}

		int aux0; float aux1;
		for(int i=0;i<N-1;i++){
			for(int j=0; j<N-1;j++){
				if(valores[j][1]<valores[j+1][1]){
					aux0 = valores[j][0];
					aux1 = valores[j][1];
					valores[j][0] = valores[j+1][0];
					valores[j][1] = valores[j+1][1];
					valores[j+1][0] = aux0;
					valores[j+1][1] = aux1;
				}
			}
		} //preordena
		

    List<Defense*>::iterator currentDefense = defenses.begin();
		float posx, posy; int row, col, i = 0;

    while(currentDefense != defenses.end()) {
			row = (int)valores[i][0]/nCellsHeight;
			col = (int)valores[i][0]%nCellsWidth;
			posx= col * cellWidth + cellWidth * 0.5f;
			posy= row * cellHeight + cellHeight * 0.5f;
			if(factibilidad(posx, posy, mapWidth,mapHeight,currentDefense,defenses,obstacles)){
				(*currentDefense)->position.x = posx;
				(*currentDefense)->position.y = posy;
				(*currentDefense)->position.z = 0;
				++currentDefense;

			}
			i++;
}

\end{lstlisting}


\end{enumerate}

Todo el material incluido en esta memoria y en los ficheros asociados es de mi autoría o ha sido facilitado por los profesores de la asignatura. Haciendo entrega de este documento confirmo que he leído la normativa de la asignatura, incluido el punto que respecta al uso de material no original.

\end{document}
